\documentclass[main.tex]{subfiles}

% The following commands are from the \textpages macro
% and are needed for the subfile properly.
% These commands are ignored when compiling main.tex
\pagestyle{empty}
\cleardoublepage
\pagestyle{myheadings}\markboth{}{}
\def\@chapapp{\protect\@textofChapter}
\def\Thechapter{\arabic{chapter}}
\pagenumbering{arabic}
% End of commands


\begin{document}

\chapter{Introduction}

\section{Background}
This template is an updated version of the original UW thesis template and is a fork of the corresponding GitHub repository: \href{https://github.com/UWIT-IAM/UWThesis}{https://github.com/UWIT-IAM/UWThesis}.
The original had not been updated in some time and was a bit cumbersome to work with so I have made some modifications to simplify it.
The primary change is that it now utilizes the \texttt{subfiles} package to enable separate compilation of all the individual components.
Additionally I've made some (personal-preference) formatting adjustments including a floating box to the reference publications(s) that a given chapter is based on.
I've tested this package on Overleaf using the \texttt{pdfLaTeX} compiler but not any other compilers or platforms.


\section{How to use this template}
Each of the main sections are grouped into their own \texttt{.tex} files which can all be compiled separately.
There are some opaque commands in the headers of the subfiles which enable the separate compilation but you only need to add the main body text where it specifies.
Note that chapter numbers and the like will not be maintained in separate compilation but will be restored when compiling the whole thesis.
To compile the full thesis simply compile the \texttt{main.tex} file.
Separate bibliographies and \texttt{.bib} files per chapter may be possible but have not been tested.
If you are adding a new chapter, simply copy the header of one of the existing chapters and reference it in \texttt{main.tex} via the \texttt{subfile} macro.
You should not need to add any text into the main file directly.
Most of the style formatting can be addressed in the header of \texttt{main.tex}.
If you need to make adjustments to the actual typesetting of the UW style file then you will have to directly modify the TeX macros in \texttt{uwthesis.cls}.



\end{document}